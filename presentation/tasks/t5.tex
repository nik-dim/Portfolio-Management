% \section{Question 5}
\subsection{Financial Metrics}    
% \textbf{Compute additional metrics for the assets such as the correlation and covariance matrices, for the entire and two subperiods (of your own choosing), if needed. Interpret your findings. Also, compute each fund’s alpha, beta, R-square. Interpret your findings from the perspective of the investor}

% \subsection{correlation and covariance}
\begin{frame}
    \frametitle{Covariance}

    \begin{block}{Definition}
        Let \( X \) and \( Y \) be two random variables. Then the covariance is a measure of the joint variability of these two random variables:

        \begin{equation*}
            \label{eq:covariance}
            cov(X,Y) = \mathbb{E}[(X-\bar{x})(Y-\bar{y})] 
        \end{equation*}
    \end{block}
    \pause
    \begin{itemize}
        \item Not so helpful! \pause \( \rightarrow \) correlation
    \end{itemize}
\end{frame}



\begin{frame}
    \frametitle{Correlation}

    \begin{block}{Definition}
        The correlation is the normalization of the covariance.
        \begin{equation*}
            \label{eq:correlation}
            \rho_{X,Y}=\frac{cov(X,Y)}{\sigma_X \cdot\sigma_Y}
        \end{equation*}
    \end{block}
    % The correlation ranges between \( -1 \) and \( 1 \) and it measures the linear dependence between two variables. 
    \pause
    \begin{equation*}
        \rho_{X,Y}\begin{cases}
            =-1, & \text{perfect decreasing (inverse) linear relationship}\\
            \in (-1,1), & \text{indicating the degree of linear dependence}\\
            =1, & \text{perfect (increasing) linear  relationship}
        \end{cases}
    \end{equation*}   
\end{frame}


\begin{frame}
    \frametitle{A closer look at correlation}

	\begin{figure}[H]
        \centering
        \includegraphics[width=\linewidth]{../report/media/positive_vs_negative_correlation.png}
        \label{fig:positive_vs_negative_correlation}
    \end{figure}
\end{frame}

\begin{frame}
	\frametitle{Correlation}
	\framesubtitle{From an investor's perspective}
	\begin{itemize}
		\item Powerful tool that measures the strength of linear relationship between the price movements of two individual securities
		\pause
		\item The total risk of two correlated assets is:
		\begin{equation*}
		\sigma^{2}=w_{x}^{2}\cdot \sigma_{x}^{2} + w_{y}^{2}\cdot \sigma_{y}^{2} + 2\cdot w_{x} \cdot w_{y} \cdot \sigma_{x} \cdot \sigma_{y} \cdot \rho_{x,y} 
		\end{equation*}
		\item Negatively correlated assets yield less risk, compared to positively correlated ones
		\pause
		\item Investors can diversify risk by including negatively correlated assets in their portfolios
	\end{itemize}
\end{frame}

% the correlation matrix is presented below:
\begin{frame}
    \frametitle{Correlation Matrix}
	\begin{figure}[H]
        \centering
        \includegraphics[width=0.8\linewidth]{../report/media/Correlation_Matrix.png}
        % \caption{The correlation matrix for our instruments.}
        \label{fig:correlation matrix}
    \end{figure}
\end{frame}

% The correlation matrix is also shown in Figure \ref{fig:correlation matrix} and in Figure \ref{fig:pairplot} another visualitization of the effect of the various values of correlation coefficients is presented:




% \subsection{beta}  
\begin{frame}
    \frametitle{Beta}
    \begin{block}{Definition}
        The beta coefficient measures the systematic risk of an individual stock compared to the market risk: 
        % The beta formula is:
        \begin{equation*}
            \label{eq:beta}
            \beta = \frac{\text{cov}(R_e,R_m)}{\text{var}(R_m)}
        \end{equation*}
    \end{block}
    \pause
    \begin{itemize}
    	\item Investors use \(\beta \) to determine the movement direction and volatility of a security in comparison with the market\pause
    	\item A high \(R^{2}\) is required for \(\beta \) to be meaningful
    \end{itemize}    

\end{frame}

% \subsection{alpha} 
\begin{frame}
    \frametitle{alpha}
    
    \begin{block}{Definition}
        Alpha is the difference between the realised returns and the expected returns:
        \begin{align*}
            \alpha &= \bar{R} - \mathbb{E} (R) 
            \\\stackrel{\text{CAPM}}{\implies}\alpha&=\bar{R} - \left\{R_{f}+\beta(\mathbb{E}  (R_{m})-R_{f})\right\}
        \end{align*} 
    \end{block}
    \pause
    \begin{itemize}
    	\item Investors use \(\alpha \) to determine whether or not a security has exceeded expectations in terms of return
    	\item Frequently used to quantify the "added" value of the manager
    \end{itemize}

\end{frame}

\begin{frame}
    \frametitle{\( R \)-squared}
    \begin{block}{Definition}    	
        \begin{equation*}
            \label{eq:r-squared}
            R^2=1-\frac{\text{Explained Variation}}{\text{Total Variation}}
        \end{equation*}
        % Denotes the percentage of a fund's movements that can be explained away by the historical changes of the benchmark index. 
    \end{block}
    \pause
    \begin{figure}[]
        \centering
        \begin{subfigure}[t]{0.5\textwidth}
            \centering
            \includegraphics[width=\textwidth]{../report/media/regression_line.png}
            \caption{Explained Variation}
            \label{fig:rsquared_explanation_regression}
        \end{subfigure}%
        ~ 
        \begin{subfigure}[t]{0.5\textwidth}
            \centering
            \includegraphics[width=\textwidth]{../report/media/mean_line.png}
            \caption{Total Variation}
            \label{fig:q2}
        \end{subfigure}
    \end{figure}
	
	\begin{itemize}
		\item \(R-squared\) measures the "fitness" of a regression model
	\end{itemize}
	
\end{frame}


% \subsection{Sharpe ratio} 
\begin{frame}
    \frametitle{Sharpe Ratio}
    \begin{block}{Definition}
    	Sharpe Ratio relates the return of an investment to its risk:
        \begin{equation*}
            \label{eq:sharpe ration}
            \text{Sharpe Ratio} = \frac{R_p-R_f}{\sigma_p}    
        \end{equation*}  
    \end{block}
	\pause
	\begin{itemize}
		\item Measures the risk-adjusted return \pause		
		\item i.e. the average return earned \textit{in excess} of the risk-free rate per unit of volatility or total risk.\pause
		\item Useful for comparing the performance of investments that have different levels of risk and return
	\end{itemize}
	
\end{frame}

\begin{frame}
    \frametitle{Financial Metrics}
    \framesubtitle{For our assets}
    
    
    \begin{center}
        % \setlength{\tabcolsep}{2pt}
        % \tiny
        % \begin{adjustwidth}{}{}
            \begin{tabular}{lccc}
                \toprule
                {} &    alpha &     beta &  R-squared \\
                \midrule
                AVGO      &  0.14213 &  1.35333 &    0.31393 \\
                NOW       &  0.24718 &  1.46614 &    0.30955 \\
                TSLA      &  0.17711 &  1.26445 &    0.14046 \\
                AMZN      &  0.32815 &  1.32199 &    0.36539 \\
                IRDM      &  0.15867 &  1.41795 &    0.20698 \\
                BR        &  0.13206 &  0.92064 &    0.37655 \\
                PYACX     & -0.00576 & -0.06792 &    0.04613 \\
                SIGAX     & -0.01215 & -0.01911 &    0.00501 \\
                gold      &  0.05087 & -0.15174 &    0.01684 \\
                palladium &  0.17817 &  0.47398 &    0.05493 \\
                \bottomrule
                \end{tabular}
                
        % \end{adjustwidth}
    \end{center}

\end{frame}
