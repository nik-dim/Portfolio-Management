\section{Question 9} 
\label{optimization}
\textbf{Measure and evaluate your overall portfolio’s performance and compare it with the passive investment strategy. In this step, you should apply EXCEL’s Solver to evaluate several possible outcomes (in terms of risk and return) and explain each outcome. In that endeavour, compute the various performance measures we have learned. Decide on the best outcome for you. Discuss.}

As we examined in Section \ref{naive_analysis}, the naive approach is not optimal when constructing a portfolio of risky assets. In this part we will examine how optimization analysis helps investors determine the optimal risky portfolio and how the optimal overall allocation between risk-free and risky assets is derived.

According to Markowitz, father of Modern Portfolio Theory, the portfolio construction process can be separated into two different independent tasks:

\begin{itemize}
	\item Determination of the optimal risky portfolio, using the investor's selected assets. This part is purely technical.
	\item Determination of the overall portfolio. This part includes the investor's decision for allocation of funds between risk-free assets and the risky portfolio. This choice depends on the investor's personal preference and his/her risk profile.
\end{itemize}

The above concept is known as the Separation Theorem. We can therefore break our final choice into two separate problems.

\subsection{Optimal Risky Portfolio}

As illustrated in Figure \ref{fig:random_portfolios_and_EF}, the Efficient Frontier consists of  \(minimum-variance\) portfolios. Therefore the optimality criterion lies in the Sharpe Ratio associated with each risky portfolio. The portfolio with the highest Sharpe Ratio offers the best \(risk-adjusted\) \(returns\) among all the available combinations in the opportunity set.

In order to find the optimal portfolio, we ought to solve the following optimization problem:

\begin{align*}
\text{max }\quad & \frac{\vec{w}^\top\cdot \mathbb{E} (\mathcal{R}) -r_f}{\sqrt{\vec{w}^\top K  \vec{w} }}\\
\text{s.t. }\quad & \mathbf{ 1 }^\top \vec{w}=1 \\
& \mathbf{ 1 }^\top \vec{w}_{\mathcal{ S } } = w_s \\
& \mathbf{ 1 }^\top \vec{w}_{\mathcal{ B } } = w_b \\
& \mathbf{ 1 }^\top \vec{w}_{\mathcal{ C } } = w_c \\
& w_s + w_b + w_c = 1 \\
& w_i \geq 0 \quad i=1,\dots,n
\end{align*}
where \( \vec{w} \) is the weight vector associated with the \( n \) assets, \( \vec{w}_{\mathcal{ S } } \), \( \vec{w}_{\mathcal{ B } } \) and \( \vec{w}_{\mathcal{ C } } \) are the weights associated with the stock, bond and commodity assets and sum up to \( w_s \), \( w_b \) and \( w_c \), respectively. Obviously, the concatenation of these vectors is the original weight vector, i.e. \( \mathcal{ S } \sqcup \mathcal{ B } \sqcup \mathcal{ C } =\{1,\cdots,n\}  \). In other words, the objective corresponds to the Sharpe Ratio and the constraints construct the feasibility region of the optimization problem given the demands dictated by the suggested allocation resulting in the weights \( w_s \), \( w_b \) and \( w_c \).

\newpage
By solving the above optimization problem the optimal weights of securities in the risky portfolio are calculated:

\begin{equation}
	% ['AVGO', 'NOW', 'TSLA', 'AMZN', 'IRDM', 'BR', 'PYACX', 'SIGAX', 'gold', 'palladium', 'S&P 500']
	\label{eq:naive weight allocation}
	\vec{w}=\begin{bmatrix}
		w_{\text{AVGO}}\\
		w_{\text{NOW}}\\
		w_{\text{TSLA}}\\
		w_{\text{AMZN}}\\
		w_{\text{IRDM}}\\
		w_{\text{BR}}\\
		w_{\text{PYACX}}\\
		w_{\text{SIGAX}}\\
		w_{\text{gold}}\\
		w_{\text{palladium}}
	\end{bmatrix}=\begin{bmatrix}
	4.0068\\8.6743\\1.0266\\38.2827\\1.32\\16.6897\\
	20.0\\0.0\\0.0\\10.0
	\end{bmatrix}\%
\end{equation}

This optimal allocation yields the following results:
\begin{center}
	%\begin{adjustwidth}{\adjust}{}
	\begin{tabular}{l|cc}
		\toprule
		{}   & Optimal Risky Portfolio & Naive Risky Portfolio \\
		\midrule
		Return (\%) & 32.7164 & 27.444\\
		Risk (\%) & 16.3185 & 15.7565\\
		Sharpe Ratio & 1.8658 & 1.5977\\   
		Beta (\( \beta \)) & 0.9067 & 0.9109\\
		\bottomrule
	\end{tabular}
	%\end{adjustwidth}
\end{center}

\vspace{\baselineskip}

The analysis showed that \texttt{gold} and \texttt{SIGAX} have zero weights. We have two choices:

\begin{itemize}
	\item Exclude these securities from the optimal portfolio, as they have no effect on its risk and return or
	\item Substitute them with other securities of the same class. 
\end{itemize}

The optimal Sharpe Ratio is also the slope of the Capital Allocation Line. The CAL of the optimal risky portfolio is tangent to the Efficient Frontier, as shown in Figure \ref{fig:final_graph}. The equation of the optimal CAL  (\( \epsilon_{opt} \)) is given below:

\begin{equation}
\label{eq:cal_opt}
\left\{\begin{array}{c}
(0,r_f) \in \epsilon_{\text{opt}}     \\
(\sigma_{\text{opt}},r_{\text{opt}}) \in \epsilon_{\text{opt}}     
\end{array}    \right\} \implies \epsilon_{opt}: y = \underbrace{\frac{r_{\text{opt}}-r_f}{\sigma_{\text{opt}}}}_{\text{max Sharpe Ratio}}\cdot x + r_f
\end{equation}

\subsection{Optimal Overall Portfolio}

The final step in the portfolio construction is to decide on the allocation between risk-free assets and the risky portfolio. This decision is based on the investor's attitude towards risk. An individual's risk profile can be quantified by the coefficient of risk aversion, denoted \(A\). Low values of \(A\) indicate risk-tolerance, while high values indicate risk-aversion. Various studies have been conducted to determine the range of \(A\) and is usually placed between 1 and 6. As young and risk-tolerant investors, we assume that our coefficient of risk-aversion is \(A=1.2\).

Let \(z\) be the proportion of funds allocated to the optimal risky portfolio. Then \((1-z)\) is the proportion allocated to risk-free assets. Investors calculate the optimal \(z\) by maximizing their utility. Expressed as a function of z, utility of the optimal overall portfolio is given by the equation below:

\begin{equation}
\label{eq:utility}
\text{Utility} = U = r_{f} + z \cdot (r_{\text{opt}} - r_{f}) - 0.05 \cdot A \cdot \sigma_{\text{opt}}^{2} \cdot z^{2}
\end{equation}

Therefore, to maximize the utility w.r.t the allocation \( z \), we set the first derivative equal to zero. The optimal \(z^{*}\) that maximizes utility is:

\begin{gather*}
\label{eq:z_optimal}
	\frac{\partial U}{\partial z} =0 \\
	\implies r_{\text{opt}}-r_f - 0.1 \cdot A \cdot \sigma^2_{\text{opt}}\cdot z=0\\
	\implies z^{*} = \cfrac{r_{\text{opt}} - r_f}{0.1 \cdot A \cdot \sigma_{\text{opt}}^2} \numberthis
\end{gather*}

After calculations, we derive that \(z^{*}=0.95\) and \(U_{max}=16.73\). By finding different levels of risk an return that yield the same value of maximum utility, we can graph the indifference curve for \(U_{max}=16.73\) and \(A=1.2\). As illustrated in Figure \ref{fig:final_graph}, the optimal overall portfolio occurs at the point where the indifference curve is tangent to the CAL of the optimal risky portfolio.

\begin{figure}[H]
	\centering
	\includegraphics[width=\linewidth]{media/Final_Graph.png}
	\captionsetup{width=0.7\linewidth}
	\caption{ After Sharpe Ratio maximization, the optimal CAL is tangent to the EF. The optimal overall portfolio for the investor occurs at the tangency point between the indifference curve and the CAL.}
	\label{fig:final_graph}
\end{figure}

\subsection{Performance Evaluation of Overall Portfolio}
After calculating \(z^{*}\), we can measure the overall portfolio's performance and compare it with the passive investment strategy. The \texttt{S\&P 500} is chosen as the market index, representing passive investing.

\paragraph*{Past Performance (1/13/2015 - 1/13/2020)} The results are presented in the table below:

\begin{center}
	%\begin{adjustwidth}{\adjust}{}
	\begin{tabular}{l|cc}
		\toprule
		{}   & Optimal Overall Portfolio & S\&P 500 \\
		\midrule
		Return (\%) & 31.1941 & 11.3118\\
		Risk (\%) & 15.5026 & 13.4138\\
		Sharpe Ratio & 1.8658 & 0.6741\\   
		Beta (\( \beta \)) & 0.8613 & 1.0 \\
		Alpha (\( \alpha \)) (\%) & 21.1362 & 0.0\\
		\bottomrule
	\end{tabular}
	%\end{adjustwidth}
\end{center}

\vspace{\baselineskip}

We can conclude that holding the overall portfolio is the best choice for us, since it not only offers superior returns compared to the market index, but does so with a higher Sharpe Ratio. The overall portfolio's \( \alpha \) was approximately 16\%, indicating that the portfolio exceeded return expectations and that some of the holding securities are under-valued.

\paragraph*{Holding Period Returns (1/14/2020 - 4/20/2020)} It is interesting to additionally calculate the Holding Period Return (HPR) of the overall portfolio and compare it to the market index, as the world economy has experienced a rocky ride in the last quarter, due to reasons partly presented in Section \ref{events}. The formula is presented below:

\begin{equation}
\label{eq:z_optimal}
\text{Holding Period Return (\%)} = \cfrac{\text{End of Period value} - \text{Initial Value}}{\text{Initial Value}} \cdot 100
\end{equation}

It is noted that all cash flows and income of the period are reflected in the ending value. The results are presented below:

\begin{center}
	%\begin{adjustwidth}{\adjust}{}
	\begin{tabular}{l|cc}
		\toprule
		{}   & Optimal Overall Portfolio & S\&P 500 \\
		\midrule
		HPR (\%) & +6.8182 & -14.7298\\
		\bottomrule
	\end{tabular}
	%\end{adjustwidth}
\end{center}

\vspace{\baselineskip}

The overall portfolio has managed to produce positive returns, due to the inclusion of both bonds and commodities that offer risk diversification, as well as the high performance of certain stocks. Therefore the overall decline observed for the market, represented by the \texttt{S\&P 500} index, was partially mitigated to our overall portfolio.





