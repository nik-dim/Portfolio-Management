\section{Question 1} 
\label{IPS}
\textbf{Decide on the allocation of your budget to each of the above instruments. Rationalize your choice based on your own investment philosophy. You should fill out a risk tolerance questionnaire and supply both the questionnaire and its suggested allocation (in the project’s Appendix). In your discussion, please explain the choice of your investment vehicles and their significance in your portfolio. Label this as your investment policy statement (IPS). }

\paragraph*{Investment Policy Statement}

We are classified as young investors, with a small initial capital available. A summary of our investment philosophy is as follows:

\paragraph*{Objectives}
\begin{itemize}
	\item Long-term growth and capital appreciation
	\item Risk-profile: Aggressive
	\item Time horizon: Greater than 10 years
	\item Short-term liquidity needs: Minimal
\end{itemize}

\paragraph*{Portfolio Selection Guidelines}
Long-term investment performance is generally determined by the underlying asset's performance. From a historical point of view, stock assets tend to achieve higher rates of return, along with greater volatility. Fixed income assets (i.e bonds) yield lower rates of return and bear less risk. 

Based on our risk tolerance profile (see Appendix) and aim for long-term capital growth, the portfolio asset allocation will be:
\begin{itemize}
	\item 70\% Stocks
	\item 20\% Bonds
	\item 10\% Commodities
\end{itemize}

The selected assets are traded on US exchanges. The individual composition of holdings will be selected from the following asset classes:

\paragraph*{Equity}
Equities are the main investment instrument of our portfolio, which will implement the required capital growth. A total of 6 stocks are selected, with the following characteristics: 
\begin{itemize}
	\item Company Size: 4 Large, 1 Mid, 1 Small
	\item Stock Type: 4 Growth, 1 Aggressive Growth, 1 High-Dividend
\end{itemize}

\paragraph*{Bonds}
Fixed income assets are of secondary importance, due to our minimal needs for generation of current income. Instead of two individual corporate bonds, 2 bond mutual funds are selected, which invest at least 85\% of their assets in corporate bonds.

\paragraph*{Commodities}
For a portfolio that has a long-term horizon, allocating a small amount to commodities offers some protection against inflation and reduces the overall portfolio risk. Commodities historically move opposite from the market and therefore yield a negative correlation regarding other selected securities. This fact, as explained in \ref{Correlation}, offers risk diversification benefits. A total of 2 precious metals are selected. 

A more elaborate analysis of the final selected assets is performed in Section 2.

\paragraph*{Risk-free Rate}
The 10-Year U.S. Treasury Bond is chosen as the risk-free asset, as it matches the time horizon of our investment.

\paragraph*{Benchmark Index}
The S\&P 500 index is used as the benchmark index of the portfolio.