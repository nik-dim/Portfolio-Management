\section{Question 6}  
\textbf{Calculate the return/risk of your risky portfolio. Explain each step in your analysis. You must use EXCEL’s mmult functions for this part of the analysis.}

Up until this point we have selected the assets that we will include in our risky portfolio and we have calculated important descriptive statistics and metrics for each individual security. In this section we will calculate the risk and return of the risky portfolio, using a vector of weights. Each security is assigned a weight, representing a portion of the available capital that is to be invested in it. 

Let \( R_i \) be the returns data sample for the instrument \( i \) and \( \mathcal{ R } = (R_1, R_2,\cdots,R_n) \) be the collection of those random variables. 
Let \( \vec{w} \) be the vector of weights that represents the allocation of the various instruments in our risky portfolio. Obviously, \( \|\vec{w}\|_1=1 \). Finally, let \( K \) be the associated covariance matrix.

Then the return of the risky portfolio is a weighted average of the expected returns:

\begin{equation}
    \label{eq:return of risky portfolio}
    R_{\text{risky portfolio}}= \vec{w}^\top\cdot \mathbb{E} (\mathcal{R}) 
\end{equation}
and the risk of the risky portfolio is measured by its standard deviation:
\begin{align}
    \label{eq:risk of risky portfolio}
    \sigma_{\text{risky portfolio}}&= \sqrt{\vec{w}^\top K  \vec{w} }\\
    &=\sqrt{\begin{bmatrix}    
        w_1 & w_2 & \dots  & w_n
    \end{bmatrix}
    \begin{bmatrix}
        \sigma_{1}^2       &   \text{cov}_{1,2}       & \dots     &   \text{cov}_{1,n}       \\
        \text{cov}_{2,1}       &   \sigma_{2}^2       & \dots     &   \text{cov}_{2,n}       \\
        \vdots          &  \vdots   &   \ddots   &   \vdots  \\   
        \text{cov}_{n,1}       &   \text{cov}_{n,2}       & \dots     &   \sigma_{n}^2      \\
    \end{bmatrix}    
    \begin{bmatrix}
        w_1     \\
        w_2     \\
        \vdots  \\
        w_n     \\
    \end{bmatrix}  
     } \\
    &=\sqrt{\sum\limits_{i=1}^{n} w_i^2\sigma_i^2 +\sum\limits_{i=1}^{n} \sum\limits_{j=1}^{n} w_iw_j \text{cov}_{i,j}   }
\end{align}

As an initial approach, available capital is allocated uniformly between assets belonging in the same class. In other words, let \( w\% \) be the percentage allocated to a class of \( n \) assets (i.e. stocks, bonds or commodities). Then, each security of this class receives a \( \frac{w\%}{n} \) allocation. As stated in the portfolio's IPS (\ref{IPS}), we have \( w_{stocks}=70\% \), \( w_{bonds}=20\% \) and \( w_{commodities}=10\% \) Hence, the weights associated with the securities are:

\begin{equation}
    % ['AVGO', 'NOW', 'TSLA', 'AMZN', 'IRDM', 'BR', 'PYACX', 'SIGAX', 'gold', 'palladium', 'S&P 500']
    \label{eq:naive weight allocation}
    \vec{w}=\begin{bmatrix}
        w_{\text{AVGO}}\\
        w_{\text{NOW}}\\
        w_{\text{TSLA}}\\
        w_{\text{AMZN}}\\
        w_{\text{IRDM}}\\
        w_{\text{BR}}\\
        w_{\text{PYACX}}\\
        w_{\text{SIGAX}}\\
        w_{\text{gold}}\\
        w_{\text{palladium}}
    \end{bmatrix}=\begin{bmatrix}
        11.67\\11.67\\11.67\\11.67\\11.67\\11.67\\
        10.0\\10.0\\5.0\\5.0
    \end{bmatrix}\%
\end{equation}

This naive allocation yields the following results:
\begin{center}
	%\begin{adjustwidth}{\adjust}{}
        \begin{tabular}{l|c}
        \toprule
        {}   & Naive Risky Portfolio \\
        \midrule
        Return (\%) & 27.444 \\
        Risk (\%) & 15.7565 \\
        Sharpe Ratio & 1.5977\\   
        \bottomrule
        \end{tabular}
    %\end{adjustwidth}
\end{center}
