\section{Question 10}  
In this project we had the opportunity to engage in the application of an investment process, in order to construct a portfolio that meets our personal goals and preferences. This process was performed end to end, from asset allocation to portfolio evaluation. Several core concepts of Modern Portfolio Theory were examined and can be summarized as follows:

\begin{itemize}
	\item Investor's risk profile assessment by utilizing risk tolerance questionnaire.
	\item Asset Allocation, Security Selection and Diversification Strategy.
	\item Technical Analysis for computing each security's statistics and other financial metrics.
	\item Risky Portfolio construction with risk and return trade-off maximization. Graphical illustration of the Efficient Frontier and the Capital Allocation Line.
	\item Overall portfolio construction that satisfies the investor's risk appetite.
	\item Portfolio evaluation based on a benchmark index.
\end{itemize}

We were able to grasp a solid understanding of the fundamental concepts associated with portfolio management and identify some key takeaways. The construction of a portfolio is based on the individual investor's time horizon (either short or long-term) and attitude towards risk. For the long-term prospect, stocks are more preferable than other financial instruments. Portfolio management requires a steady monitoring of holding assets and tracking of economic or other events that influence the portfolio. Finally, portfolio evaluation requires identification of a suitable benchmark that reflects the composition of assets in the portfolio. In this project, the \texttt{S\&P500} was chosen as the benchmark. However, since the portfolio also includes bonds and commodities, a better practice would be to construct a blended benchmark consisting of both the \texttt{S\&P500} index and the \texttt{Bloomberg Barclays US Aggregate Bond} index and matching the allocation of each asset class. Therefore, each asset class will be compared to its corresponding index and the evaluation results will be more accurate, allowing the investor to draw better decisions for his/her portfolio.