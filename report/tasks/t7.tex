\section{Question 7}  
\textbf{Derive and graph the Capital Allocation Line. Graph the Efficient Frontier with your available investment instruments (assets) and superimpose your CAL. Discuss the various options you may have and finalize your optimal point.}

By taking various random weight allocations we are able to create portfolios with different risks and returns. The optimality criterion lies in the Sharpe Ratio associated with each risky portfolio. The optimal risky portfolio is characterized by the highest Sharpe ratio.

\begin{figure}[H]
    \centering
    \includegraphics[width=\linewidth]{media/Random_Portfolios_and_EF.png}
    \captionsetup{width=0.7\linewidth}
    \caption{\( n=50,000 \) random portfolios are created and plotted in a common graph. The optimal risky portfolio and the Efficient Frontier also appear.}
    \label{fig:random_portfolios_and_EF}
\end{figure}

The above graph is conducive with the theoretical one with respect to its shape. By creating the random portfolios with a large enough sample size \( (n=50,000)\), the range of annualized returns and volatilities is known. Yo find the Efficient Frontier an optimization problem must be solved. To be more specific, we seek to maximize the Sharpe Ratio given a fixed annualized portfolio return. By solving numerous such optimization problems, points on the Efficient Frontier are obtained and graphed. 

In order to find the optimal portfolio, we ought to solve the following optimization problem:

\begin{align*}
    \text{max }\quad & \frac{\vec{w}^\top\cdot \mathbb{E} (\mathcal{R}) -r_f}{\sqrt{\vec{w}^\top K  \vec{w} }}\\
    \text{s.t. }\quad & \mathbf{ 1 }^\top \vec{w}=1 \\
    & \mathbf{ 1 }^\top \vec{w}_{\mathcal{ S } } = w_s \\
    & \mathbf{ 1 }^\top \vec{w}_{\mathcal{ B } } = w_b \\
    & \mathbf{ 1 }^\top \vec{w}_{\mathcal{ C } } = w_c \\
    & w_s + w_b + w_c = 1 \\
    & w_i \geq 0 \quad i=1,\dots,n
\end{align*}
where \( \vec{w} \) is the weight vector associated with the \( n \) assets, \( \vec{w}_{\mathcal{ S } } \), \( \vec{w}_{\mathcal{ B } } \) and \( \vec{w}_{\mathcal{ C } } \) are the weights associated with the stock, bond and commodity assets and sum up to \( w_s \), \( w_b \) and \( w_c \), respectively. Obviously, the concatenation of these vectors is the original weight vector, i.e. \( \mathcal{ S } \sqcup \mathcal{ B } \sqcup \mathcal{ C } =\{1,\cdots,n\}  \). In other words, the objective corresponds to the Sharpe Ratio and the constraints construct the feasibility region of the optimization problem given the demands dictated by the suggested allocation resulting in the weights \( w_s \), \( w_b \) and \( w_c \).

The Capital Allocation Line or CAL is the line that connects the Optimal Portfolio with the risk-free portfolio. These two points suffice in deriving the equation of the CAL  \( \epsilon_{CAL} \):
\begin{equation}
    \label{eq:cal}
    \left\{\begin{array}{c}
        (0,r_f) \in \epsilon_{\text{CAL}}     \\
        (\sigma_{\text{OPT}},r_{\text{OPT}}) \in \epsilon_{\text{CAL}}     
    \end{array}    \right\} \implies \epsilon_{CAL}: y = \underbrace{\frac{r_{\text{OPT}}-r_f}{\sigma_{\text{OPT}}}}_{\text{max Sharpe ratio}}\cdot x + r_f
\end{equation}

Thus, we are able to plot the CAL alongside the efficient frontier and the random portfolios in figure \ref{fig:final_graph}. Solving this optimization problem yields the following results:

\begin{todo}{EF results}{todo:EF results}
  \begin{equation}
      \label{eq:optimal weights}
      \vec{w} = [\cdots]
  \end{equation}
\end{todo}


\begin{figure}[]
    \centering
    \includegraphics[width=\linewidth]{media/Final_Graph.png}
    \captionsetup{width=0.7\linewidth}
    \caption{The CAL passes through the risk-free and risky portfolio. The points of the line between these values are a linear combination of the two alternatives. The individual assets are presented. The naive CAL is also depicted. The benefits of the optimization are apparent.}
    \label{fig:final_graph}
\end{figure}

\begin{todo}{Observations}{todo:Observations}
\paragraph*{Observations}
talk about figure \ref{fig:final_graph}.
\end{todo}
