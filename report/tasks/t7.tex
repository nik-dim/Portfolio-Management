\section{Question 7} 
\label{naive_analysis}
\textbf{Derive and graph the Capital Allocation Line. Graph the Efficient Frontier with your available investment instruments (assets) and superimpose your CAL. Discuss the various options you may have and finalize your optimal point.}

By taking various random weight allocations we are able to create portfolios with different risks and returns. If we choose a large enough sample size \( (n=2,000,000)\), this will be a fair representation of the all the risky portfolios we can construct (portfolio universe). The results of the simulation are illustrated in Figure \ref{fig:random_portfolios}.

\begin{figure}[H]
	\centering
	\includegraphics[width=\linewidth]{media/Random_Portfolios.png}
	\captionsetup{width=0.7\linewidth}
	\caption{\( n=2,000,000 \) random portfolios are created and plotted in a common graph. The portfolio with the highest Sharpe Ratio from the simulation is annotated.}
	\label{fig:random_portfolios}
\end{figure}

The above graph is conducive with the theoretical one, with respect to its shape. By using a large sample size, the range of annualized returns and volatilities is known. To find the Efficient Frontier an optimization problem must be solved. The Efficient Frontier is essentially the \( Minimum-Variance\) Frontier, therefore for any given portfolio return we seek to minimize the portfolio volatility. By solving numerous such optimization problems, points on the Efficient Frontier are obtained and graphed. The results are illustrated in Figure \ref{fig:random_portfolios_and_EF}.

\begin{figure}[H]
    \centering
    \includegraphics[width=\linewidth]{media/Random_Portfolios_and_EF.png}
    \captionsetup{width=0.7\linewidth}
    \caption{The Efficient Frontier.}
    \label{fig:random_portfolios_and_EF}
\end{figure}

The Capital Allocation Line (CAL) is a linear representation of all possible combinations of risk-free and risky assets. It aids investors in deciding how to split their available capital between risk-free assets and the risky portfolio. For the naive portfolio in question, the CAL equation (\( \epsilon_{naive} \)) is given below:

\begin{equation}
\label{eq:cal_naive}
\left\{\begin{array}{c}
(0,r_f) \in \epsilon_{\text{naive}}     \\
(\sigma_{\text{naive}},r_{\text{naive}}) \in \epsilon_{\text{naive}}     
\end{array}    \right\} \implies \epsilon_{naive}: y = \underbrace{\frac{r_{\text{naive}}-r_f}{\sigma_{\text{naive}}}}_{\text{naive Sharpe Ratio}}\cdot x + r_f
\end{equation}

As illustrated in Figure \ref{fig:Naive_CAL_and_EF}, the naive risky portfolio of uniform weights does not plot on the EF, therefore is not the optimal point. We know the following from our "Foundations of Investments" course:

\begin{itemize}
	\item The optimal risky portfolio should plot on the EF.
	\item The optimal risky portfolio has the highest Sharpe between other risky portfolios on the EF.
	\item The CAL of the optimal risky portfolio is tangent to the EF.
\end{itemize}

Based on the above, an optimization problem that maximizes the Sharpe Ratio must be solved, in order to find the optimal vector of weights. This analysis is performed in  Section \ref{optimization}.

\begin{figure}[H]
	\centering
	\includegraphics[width=\linewidth]{media/Naive_CAL_and_EF.png}
	\captionsetup{width=0.7\linewidth}
	\caption{CAL of the naive portfolio and EF. The naive portfolio does not plot on the EF and does not have the maximum Sharpe Ratio, therefore the naive weight allocation to risky assets is not the optimal.}
	\label{fig:Naive_CAL_and_EF}
\end{figure}

